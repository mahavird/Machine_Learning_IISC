
% Default to the notebook output style

    


% Inherit from the specified cell style.




    
\documentclass[11pt]{article}

    
    
    \usepackage[T1]{fontenc}
    % Nicer default font (+ math font) than Computer Modern for most use cases
    \usepackage{mathpazo}

    % Basic figure setup, for now with no caption control since it's done
    % automatically by Pandoc (which extracts ![](path) syntax from Markdown).
    \usepackage{graphicx}
    % We will generate all images so they have a width \maxwidth. This means
    % that they will get their normal width if they fit onto the page, but
    % are scaled down if they would overflow the margins.
    \makeatletter
    \def\maxwidth{\ifdim\Gin@nat@width>\linewidth\linewidth
    \else\Gin@nat@width\fi}
    \makeatother
    \let\Oldincludegraphics\includegraphics
    % Set max figure width to be 80% of text width, for now hardcoded.
    \renewcommand{\includegraphics}[1]{\Oldincludegraphics[width=.8\maxwidth]{#1}}
    % Ensure that by default, figures have no caption (until we provide a
    % proper Figure object with a Caption API and a way to capture that
    % in the conversion process - todo).
    \usepackage{caption}
    \DeclareCaptionLabelFormat{nolabel}{}
    \captionsetup{labelformat=nolabel}

    \usepackage{adjustbox} % Used to constrain images to a maximum size 
    \usepackage{xcolor} % Allow colors to be defined
    \usepackage{enumerate} % Needed for markdown enumerations to work
    \usepackage{geometry} % Used to adjust the document margins
    \usepackage{amsmath} % Equations
    \usepackage{amssymb} % Equations
    \usepackage{textcomp} % defines textquotesingle
    % Hack from http://tex.stackexchange.com/a/47451/13684:
    \AtBeginDocument{%
        \def\PYZsq{\textquotesingle}% Upright quotes in Pygmentized code
    }
    \usepackage{upquote} % Upright quotes for verbatim code
    \usepackage{eurosym} % defines \euro
    \usepackage[mathletters]{ucs} % Extended unicode (utf-8) support
    \usepackage[utf8x]{inputenc} % Allow utf-8 characters in the tex document
    \usepackage{fancyvrb} % verbatim replacement that allows latex
    \usepackage{grffile} % extends the file name processing of package graphics 
                         % to support a larger range 
    % The hyperref package gives us a pdf with properly built
    % internal navigation ('pdf bookmarks' for the table of contents,
    % internal cross-reference links, web links for URLs, etc.)
    \usepackage{hyperref}
    \usepackage{longtable} % longtable support required by pandoc >1.10
    \usepackage{booktabs}  % table support for pandoc > 1.12.2
    \usepackage[inline]{enumitem} % IRkernel/repr support (it uses the enumerate* environment)
    \usepackage[normalem]{ulem} % ulem is needed to support strikethroughs (\sout)
                                % normalem makes italics be italics, not underlines
    

    
    
    % Colors for the hyperref package
    \definecolor{urlcolor}{rgb}{0,.145,.698}
    \definecolor{linkcolor}{rgb}{.71,0.21,0.01}
    \definecolor{citecolor}{rgb}{.12,.54,.11}

    % ANSI colors
    \definecolor{ansi-black}{HTML}{3E424D}
    \definecolor{ansi-black-intense}{HTML}{282C36}
    \definecolor{ansi-red}{HTML}{E75C58}
    \definecolor{ansi-red-intense}{HTML}{B22B31}
    \definecolor{ansi-green}{HTML}{00A250}
    \definecolor{ansi-green-intense}{HTML}{007427}
    \definecolor{ansi-yellow}{HTML}{DDB62B}
    \definecolor{ansi-yellow-intense}{HTML}{B27D12}
    \definecolor{ansi-blue}{HTML}{208FFB}
    \definecolor{ansi-blue-intense}{HTML}{0065CA}
    \definecolor{ansi-magenta}{HTML}{D160C4}
    \definecolor{ansi-magenta-intense}{HTML}{A03196}
    \definecolor{ansi-cyan}{HTML}{60C6C8}
    \definecolor{ansi-cyan-intense}{HTML}{258F8F}
    \definecolor{ansi-white}{HTML}{C5C1B4}
    \definecolor{ansi-white-intense}{HTML}{A1A6B2}

    % commands and environments needed by pandoc snippets
    % extracted from the output of `pandoc -s`
    \providecommand{\tightlist}{%
      \setlength{\itemsep}{0pt}\setlength{\parskip}{0pt}}
    \DefineVerbatimEnvironment{Highlighting}{Verbatim}{commandchars=\\\{\}}
    % Add ',fontsize=\small' for more characters per line
    \newenvironment{Shaded}{}{}
    \newcommand{\KeywordTok}[1]{\textcolor[rgb]{0.00,0.44,0.13}{\textbf{{#1}}}}
    \newcommand{\DataTypeTok}[1]{\textcolor[rgb]{0.56,0.13,0.00}{{#1}}}
    \newcommand{\DecValTok}[1]{\textcolor[rgb]{0.25,0.63,0.44}{{#1}}}
    \newcommand{\BaseNTok}[1]{\textcolor[rgb]{0.25,0.63,0.44}{{#1}}}
    \newcommand{\FloatTok}[1]{\textcolor[rgb]{0.25,0.63,0.44}{{#1}}}
    \newcommand{\CharTok}[1]{\textcolor[rgb]{0.25,0.44,0.63}{{#1}}}
    \newcommand{\StringTok}[1]{\textcolor[rgb]{0.25,0.44,0.63}{{#1}}}
    \newcommand{\CommentTok}[1]{\textcolor[rgb]{0.38,0.63,0.69}{\textit{{#1}}}}
    \newcommand{\OtherTok}[1]{\textcolor[rgb]{0.00,0.44,0.13}{{#1}}}
    \newcommand{\AlertTok}[1]{\textcolor[rgb]{1.00,0.00,0.00}{\textbf{{#1}}}}
    \newcommand{\FunctionTok}[1]{\textcolor[rgb]{0.02,0.16,0.49}{{#1}}}
    \newcommand{\RegionMarkerTok}[1]{{#1}}
    \newcommand{\ErrorTok}[1]{\textcolor[rgb]{1.00,0.00,0.00}{\textbf{{#1}}}}
    \newcommand{\NormalTok}[1]{{#1}}
    
    % Additional commands for more recent versions of Pandoc
    \newcommand{\ConstantTok}[1]{\textcolor[rgb]{0.53,0.00,0.00}{{#1}}}
    \newcommand{\SpecialCharTok}[1]{\textcolor[rgb]{0.25,0.44,0.63}{{#1}}}
    \newcommand{\VerbatimStringTok}[1]{\textcolor[rgb]{0.25,0.44,0.63}{{#1}}}
    \newcommand{\SpecialStringTok}[1]{\textcolor[rgb]{0.73,0.40,0.53}{{#1}}}
    \newcommand{\ImportTok}[1]{{#1}}
    \newcommand{\DocumentationTok}[1]{\textcolor[rgb]{0.73,0.13,0.13}{\textit{{#1}}}}
    \newcommand{\AnnotationTok}[1]{\textcolor[rgb]{0.38,0.63,0.69}{\textbf{\textit{{#1}}}}}
    \newcommand{\CommentVarTok}[1]{\textcolor[rgb]{0.38,0.63,0.69}{\textbf{\textit{{#1}}}}}
    \newcommand{\VariableTok}[1]{\textcolor[rgb]{0.10,0.09,0.49}{{#1}}}
    \newcommand{\ControlFlowTok}[1]{\textcolor[rgb]{0.00,0.44,0.13}{\textbf{{#1}}}}
    \newcommand{\OperatorTok}[1]{\textcolor[rgb]{0.40,0.40,0.40}{{#1}}}
    \newcommand{\BuiltInTok}[1]{{#1}}
    \newcommand{\ExtensionTok}[1]{{#1}}
    \newcommand{\PreprocessorTok}[1]{\textcolor[rgb]{0.74,0.48,0.00}{{#1}}}
    \newcommand{\AttributeTok}[1]{\textcolor[rgb]{0.49,0.56,0.16}{{#1}}}
    \newcommand{\InformationTok}[1]{\textcolor[rgb]{0.38,0.63,0.69}{\textbf{\textit{{#1}}}}}
    \newcommand{\WarningTok}[1]{\textcolor[rgb]{0.38,0.63,0.69}{\textbf{\textit{{#1}}}}}
    
    
    % Define a nice break command that doesn't care if a line doesn't already
    % exist.
    \def\br{\hspace*{\fill} \\* }
    % Math Jax compatability definitions
    \def\gt{>}
    \def\lt{<}
    % Document parameters
    \title{MNIST\_LogisticRegression}
    
    
    

    % Pygments definitions
    
\makeatletter
\def\PY@reset{\let\PY@it=\relax \let\PY@bf=\relax%
    \let\PY@ul=\relax \let\PY@tc=\relax%
    \let\PY@bc=\relax \let\PY@ff=\relax}
\def\PY@tok#1{\csname PY@tok@#1\endcsname}
\def\PY@toks#1+{\ifx\relax#1\empty\else%
    \PY@tok{#1}\expandafter\PY@toks\fi}
\def\PY@do#1{\PY@bc{\PY@tc{\PY@ul{%
    \PY@it{\PY@bf{\PY@ff{#1}}}}}}}
\def\PY#1#2{\PY@reset\PY@toks#1+\relax+\PY@do{#2}}

\expandafter\def\csname PY@tok@gd\endcsname{\def\PY@tc##1{\textcolor[rgb]{0.63,0.00,0.00}{##1}}}
\expandafter\def\csname PY@tok@gu\endcsname{\let\PY@bf=\textbf\def\PY@tc##1{\textcolor[rgb]{0.50,0.00,0.50}{##1}}}
\expandafter\def\csname PY@tok@gt\endcsname{\def\PY@tc##1{\textcolor[rgb]{0.00,0.27,0.87}{##1}}}
\expandafter\def\csname PY@tok@gs\endcsname{\let\PY@bf=\textbf}
\expandafter\def\csname PY@tok@gr\endcsname{\def\PY@tc##1{\textcolor[rgb]{1.00,0.00,0.00}{##1}}}
\expandafter\def\csname PY@tok@cm\endcsname{\let\PY@it=\textit\def\PY@tc##1{\textcolor[rgb]{0.25,0.50,0.50}{##1}}}
\expandafter\def\csname PY@tok@vg\endcsname{\def\PY@tc##1{\textcolor[rgb]{0.10,0.09,0.49}{##1}}}
\expandafter\def\csname PY@tok@vi\endcsname{\def\PY@tc##1{\textcolor[rgb]{0.10,0.09,0.49}{##1}}}
\expandafter\def\csname PY@tok@vm\endcsname{\def\PY@tc##1{\textcolor[rgb]{0.10,0.09,0.49}{##1}}}
\expandafter\def\csname PY@tok@mh\endcsname{\def\PY@tc##1{\textcolor[rgb]{0.40,0.40,0.40}{##1}}}
\expandafter\def\csname PY@tok@cs\endcsname{\let\PY@it=\textit\def\PY@tc##1{\textcolor[rgb]{0.25,0.50,0.50}{##1}}}
\expandafter\def\csname PY@tok@ge\endcsname{\let\PY@it=\textit}
\expandafter\def\csname PY@tok@vc\endcsname{\def\PY@tc##1{\textcolor[rgb]{0.10,0.09,0.49}{##1}}}
\expandafter\def\csname PY@tok@il\endcsname{\def\PY@tc##1{\textcolor[rgb]{0.40,0.40,0.40}{##1}}}
\expandafter\def\csname PY@tok@go\endcsname{\def\PY@tc##1{\textcolor[rgb]{0.53,0.53,0.53}{##1}}}
\expandafter\def\csname PY@tok@cp\endcsname{\def\PY@tc##1{\textcolor[rgb]{0.74,0.48,0.00}{##1}}}
\expandafter\def\csname PY@tok@gi\endcsname{\def\PY@tc##1{\textcolor[rgb]{0.00,0.63,0.00}{##1}}}
\expandafter\def\csname PY@tok@gh\endcsname{\let\PY@bf=\textbf\def\PY@tc##1{\textcolor[rgb]{0.00,0.00,0.50}{##1}}}
\expandafter\def\csname PY@tok@ni\endcsname{\let\PY@bf=\textbf\def\PY@tc##1{\textcolor[rgb]{0.60,0.60,0.60}{##1}}}
\expandafter\def\csname PY@tok@nl\endcsname{\def\PY@tc##1{\textcolor[rgb]{0.63,0.63,0.00}{##1}}}
\expandafter\def\csname PY@tok@nn\endcsname{\let\PY@bf=\textbf\def\PY@tc##1{\textcolor[rgb]{0.00,0.00,1.00}{##1}}}
\expandafter\def\csname PY@tok@no\endcsname{\def\PY@tc##1{\textcolor[rgb]{0.53,0.00,0.00}{##1}}}
\expandafter\def\csname PY@tok@na\endcsname{\def\PY@tc##1{\textcolor[rgb]{0.49,0.56,0.16}{##1}}}
\expandafter\def\csname PY@tok@nb\endcsname{\def\PY@tc##1{\textcolor[rgb]{0.00,0.50,0.00}{##1}}}
\expandafter\def\csname PY@tok@nc\endcsname{\let\PY@bf=\textbf\def\PY@tc##1{\textcolor[rgb]{0.00,0.00,1.00}{##1}}}
\expandafter\def\csname PY@tok@nd\endcsname{\def\PY@tc##1{\textcolor[rgb]{0.67,0.13,1.00}{##1}}}
\expandafter\def\csname PY@tok@ne\endcsname{\let\PY@bf=\textbf\def\PY@tc##1{\textcolor[rgb]{0.82,0.25,0.23}{##1}}}
\expandafter\def\csname PY@tok@nf\endcsname{\def\PY@tc##1{\textcolor[rgb]{0.00,0.00,1.00}{##1}}}
\expandafter\def\csname PY@tok@si\endcsname{\let\PY@bf=\textbf\def\PY@tc##1{\textcolor[rgb]{0.73,0.40,0.53}{##1}}}
\expandafter\def\csname PY@tok@s2\endcsname{\def\PY@tc##1{\textcolor[rgb]{0.73,0.13,0.13}{##1}}}
\expandafter\def\csname PY@tok@nt\endcsname{\let\PY@bf=\textbf\def\PY@tc##1{\textcolor[rgb]{0.00,0.50,0.00}{##1}}}
\expandafter\def\csname PY@tok@nv\endcsname{\def\PY@tc##1{\textcolor[rgb]{0.10,0.09,0.49}{##1}}}
\expandafter\def\csname PY@tok@s1\endcsname{\def\PY@tc##1{\textcolor[rgb]{0.73,0.13,0.13}{##1}}}
\expandafter\def\csname PY@tok@dl\endcsname{\def\PY@tc##1{\textcolor[rgb]{0.73,0.13,0.13}{##1}}}
\expandafter\def\csname PY@tok@ch\endcsname{\let\PY@it=\textit\def\PY@tc##1{\textcolor[rgb]{0.25,0.50,0.50}{##1}}}
\expandafter\def\csname PY@tok@m\endcsname{\def\PY@tc##1{\textcolor[rgb]{0.40,0.40,0.40}{##1}}}
\expandafter\def\csname PY@tok@gp\endcsname{\let\PY@bf=\textbf\def\PY@tc##1{\textcolor[rgb]{0.00,0.00,0.50}{##1}}}
\expandafter\def\csname PY@tok@sh\endcsname{\def\PY@tc##1{\textcolor[rgb]{0.73,0.13,0.13}{##1}}}
\expandafter\def\csname PY@tok@ow\endcsname{\let\PY@bf=\textbf\def\PY@tc##1{\textcolor[rgb]{0.67,0.13,1.00}{##1}}}
\expandafter\def\csname PY@tok@sx\endcsname{\def\PY@tc##1{\textcolor[rgb]{0.00,0.50,0.00}{##1}}}
\expandafter\def\csname PY@tok@bp\endcsname{\def\PY@tc##1{\textcolor[rgb]{0.00,0.50,0.00}{##1}}}
\expandafter\def\csname PY@tok@c1\endcsname{\let\PY@it=\textit\def\PY@tc##1{\textcolor[rgb]{0.25,0.50,0.50}{##1}}}
\expandafter\def\csname PY@tok@fm\endcsname{\def\PY@tc##1{\textcolor[rgb]{0.00,0.00,1.00}{##1}}}
\expandafter\def\csname PY@tok@o\endcsname{\def\PY@tc##1{\textcolor[rgb]{0.40,0.40,0.40}{##1}}}
\expandafter\def\csname PY@tok@kc\endcsname{\let\PY@bf=\textbf\def\PY@tc##1{\textcolor[rgb]{0.00,0.50,0.00}{##1}}}
\expandafter\def\csname PY@tok@c\endcsname{\let\PY@it=\textit\def\PY@tc##1{\textcolor[rgb]{0.25,0.50,0.50}{##1}}}
\expandafter\def\csname PY@tok@mf\endcsname{\def\PY@tc##1{\textcolor[rgb]{0.40,0.40,0.40}{##1}}}
\expandafter\def\csname PY@tok@err\endcsname{\def\PY@bc##1{\setlength{\fboxsep}{0pt}\fcolorbox[rgb]{1.00,0.00,0.00}{1,1,1}{\strut ##1}}}
\expandafter\def\csname PY@tok@mb\endcsname{\def\PY@tc##1{\textcolor[rgb]{0.40,0.40,0.40}{##1}}}
\expandafter\def\csname PY@tok@ss\endcsname{\def\PY@tc##1{\textcolor[rgb]{0.10,0.09,0.49}{##1}}}
\expandafter\def\csname PY@tok@sr\endcsname{\def\PY@tc##1{\textcolor[rgb]{0.73,0.40,0.53}{##1}}}
\expandafter\def\csname PY@tok@mo\endcsname{\def\PY@tc##1{\textcolor[rgb]{0.40,0.40,0.40}{##1}}}
\expandafter\def\csname PY@tok@kd\endcsname{\let\PY@bf=\textbf\def\PY@tc##1{\textcolor[rgb]{0.00,0.50,0.00}{##1}}}
\expandafter\def\csname PY@tok@mi\endcsname{\def\PY@tc##1{\textcolor[rgb]{0.40,0.40,0.40}{##1}}}
\expandafter\def\csname PY@tok@kn\endcsname{\let\PY@bf=\textbf\def\PY@tc##1{\textcolor[rgb]{0.00,0.50,0.00}{##1}}}
\expandafter\def\csname PY@tok@cpf\endcsname{\let\PY@it=\textit\def\PY@tc##1{\textcolor[rgb]{0.25,0.50,0.50}{##1}}}
\expandafter\def\csname PY@tok@kr\endcsname{\let\PY@bf=\textbf\def\PY@tc##1{\textcolor[rgb]{0.00,0.50,0.00}{##1}}}
\expandafter\def\csname PY@tok@s\endcsname{\def\PY@tc##1{\textcolor[rgb]{0.73,0.13,0.13}{##1}}}
\expandafter\def\csname PY@tok@kp\endcsname{\def\PY@tc##1{\textcolor[rgb]{0.00,0.50,0.00}{##1}}}
\expandafter\def\csname PY@tok@w\endcsname{\def\PY@tc##1{\textcolor[rgb]{0.73,0.73,0.73}{##1}}}
\expandafter\def\csname PY@tok@kt\endcsname{\def\PY@tc##1{\textcolor[rgb]{0.69,0.00,0.25}{##1}}}
\expandafter\def\csname PY@tok@sc\endcsname{\def\PY@tc##1{\textcolor[rgb]{0.73,0.13,0.13}{##1}}}
\expandafter\def\csname PY@tok@sb\endcsname{\def\PY@tc##1{\textcolor[rgb]{0.73,0.13,0.13}{##1}}}
\expandafter\def\csname PY@tok@sa\endcsname{\def\PY@tc##1{\textcolor[rgb]{0.73,0.13,0.13}{##1}}}
\expandafter\def\csname PY@tok@k\endcsname{\let\PY@bf=\textbf\def\PY@tc##1{\textcolor[rgb]{0.00,0.50,0.00}{##1}}}
\expandafter\def\csname PY@tok@se\endcsname{\let\PY@bf=\textbf\def\PY@tc##1{\textcolor[rgb]{0.73,0.40,0.13}{##1}}}
\expandafter\def\csname PY@tok@sd\endcsname{\let\PY@it=\textit\def\PY@tc##1{\textcolor[rgb]{0.73,0.13,0.13}{##1}}}

\def\PYZbs{\char`\\}
\def\PYZus{\char`\_}
\def\PYZob{\char`\{}
\def\PYZcb{\char`\}}
\def\PYZca{\char`\^}
\def\PYZam{\char`\&}
\def\PYZlt{\char`\<}
\def\PYZgt{\char`\>}
\def\PYZsh{\char`\#}
\def\PYZpc{\char`\%}
\def\PYZdl{\char`\$}
\def\PYZhy{\char`\-}
\def\PYZsq{\char`\'}
\def\PYZdq{\char`\"}
\def\PYZti{\char`\~}
% for compatibility with earlier versions
\def\PYZat{@}
\def\PYZlb{[}
\def\PYZrb{]}
\makeatother


    % Exact colors from NB
    \definecolor{incolor}{rgb}{0.0, 0.0, 0.5}
    \definecolor{outcolor}{rgb}{0.545, 0.0, 0.0}



    
    % Prevent overflowing lines due to hard-to-break entities
    \sloppy 
    % Setup hyperref package
    \hypersetup{
      breaklinks=true,  % so long urls are correctly broken across lines
      colorlinks=true,
      urlcolor=urlcolor,
      linkcolor=linkcolor,
      citecolor=citecolor,
      }
    % Slightly bigger margins than the latex defaults
    
    \geometry{verbose,tmargin=1in,bmargin=1in,lmargin=1in,rmargin=1in}
    
    

    \begin{document}
    
    
    \maketitle
    
    

    
    Logistic Regression (MNIST)

    The MNIST database of handwritten digits, available from this page, has
a training set of 60,000 examples, and a test set of 10,000 examples. It
is a subset of a larger set available from NIST. The digits have been
size-normalized and centered in a fixed-size image. It is a good
database for people who want to try learning techniques and pattern
recognition methods on real-world data while spending minimal efforts on
preprocessing and formatting.

    The MNIST database of handwritten digits is available on the following
website: \href{http://yann.lecun.com/exdb/mnist/}{MNIST Dataset}

    Four Files are available on this site:

    \href{http://yann.lecun.com/exdb/mnist/train-images-idx3-ubyte.gz}{train-images-idx3-ubyte.gz:
training set images (9912422 bytes)}
\href{http://yann.lecun.com/exdb/mnist/train-labels-idx1-ubyte.gz}{train-labels-idx1-ubyte.gz:
training set labels (28881 bytes)}
\href{http://yann.lecun.com/exdb/mnist/t10k-images-idx3-ubyte.gz}{t10k-images-idx3-ubyte.gz:
test set images (1648877 bytes)}
\href{http://yann.lecun.com/exdb/mnist/t10k-labels-idx1-ubyte.gz}{t10k-labels-idx1-ubyte.gz:
test set labels (4542 bytes)}

    \begin{Verbatim}[commandchars=\\\{\}]
{\color{incolor}In [{\color{incolor}19}]:} \PY{k+kn}{import} \PY{n+nn}{numpy} \PY{k+kn}{as} \PY{n+nn}{np} 
         \PY{k+kn}{import} \PY{n+nn}{matplotlib.pyplot} \PY{k+kn}{as} \PY{n+nn}{plt}
         
         \PY{c+c1}{\PYZsh{} Used for Confusion Matrix}
         \PY{k+kn}{from} \PY{n+nn}{sklearn} \PY{k+kn}{import} \PY{n}{metrics}
         \PY{k+kn}{import} \PY{n+nn}{seaborn} \PY{k+kn}{as} \PY{n+nn}{sns}
         
         \PY{c+c1}{\PYZsh{} Used for Loading MNIST}
         \PY{k+kn}{from} \PY{n+nn}{struct} \PY{k+kn}{import} \PY{n}{unpack}
         
         \PY{k+kn}{import} \PY{n+nn}{wget}
         \PY{k+kn}{import} \PY{n+nn}{os}
         \PY{o}{\PYZpc{}}\PY{k}{matplotlib} inline
\end{Verbatim}


    You can download the data via command line (you can see this on the
youtube video) or you can get them from the website or my github.

    \subsection{Downloading MNIST Dataset}\label{downloading-mnist-dataset}

    \begin{Verbatim}[commandchars=\\\{\}]
{\color{incolor}In [{\color{incolor}21}]:} \PY{n}{wget}\PY{o}{.}\PY{n}{download}\PY{p}{(}\PY{l+s+s2}{\PYZdq{}}\PY{l+s+s2}{http://yann.lecun.com/exdb/mnist/train\PYZhy{}images\PYZhy{}idx3\PYZhy{}ubyte.gz}\PY{l+s+s2}{\PYZdq{}}\PY{p}{)}
         \PY{n}{os}\PY{o}{.}\PY{n}{system} \PY{p}{(}\PY{l+s+s2}{\PYZdq{}}\PY{l+s+s2}{mv train\PYZhy{}images\PYZhy{}idx3\PYZhy{}ubyte.gz data/}\PY{l+s+s2}{\PYZdq{}}\PY{p}{)}
\end{Verbatim}


\begin{Verbatim}[commandchars=\\\{\}]
{\color{outcolor}Out[{\color{outcolor}21}]:} 0
\end{Verbatim}
            
    \begin{Verbatim}[commandchars=\\\{\}]
{\color{incolor}In [{\color{incolor}22}]:} \PY{n}{wget}\PY{o}{.}\PY{n}{download}\PY{p}{(}\PY{l+s+s2}{\PYZdq{}}\PY{l+s+s2}{http://yann.lecun.com/exdb/mnist/train\PYZhy{}labels\PYZhy{}idx1\PYZhy{}ubyte.gz}\PY{l+s+s2}{\PYZdq{}}\PY{p}{)}
         \PY{n}{os}\PY{o}{.}\PY{n}{system} \PY{p}{(}\PY{l+s+s2}{\PYZdq{}}\PY{l+s+s2}{mv train\PYZhy{}labels\PYZhy{}idx1\PYZhy{}ubyte.gz data/}\PY{l+s+s2}{\PYZdq{}}\PY{p}{)}
\end{Verbatim}


\begin{Verbatim}[commandchars=\\\{\}]
{\color{outcolor}Out[{\color{outcolor}22}]:} 0
\end{Verbatim}
            
    \begin{Verbatim}[commandchars=\\\{\}]
{\color{incolor}In [{\color{incolor}23}]:} \PY{c+c1}{\PYZsh{} !wget \PYZhy{}O data/t10k\PYZhy{}images\PYZhy{}idx3\PYZhy{}ubyte.gz http://yann.lecun.com/exdb/mnist/t10k\PYZhy{}images\PYZhy{}idx3\PYZhy{}ubyte.gz}
         \PY{n}{wget}\PY{o}{.}\PY{n}{download}\PY{p}{(}\PY{l+s+s2}{\PYZdq{}}\PY{l+s+s2}{http://yann.lecun.com/exdb/mnist/t10k\PYZhy{}images\PYZhy{}idx3\PYZhy{}ubyte.gz}\PY{l+s+s2}{\PYZdq{}}\PY{p}{)}
         \PY{n}{os}\PY{o}{.}\PY{n}{system} \PY{p}{(}\PY{l+s+s2}{\PYZdq{}}\PY{l+s+s2}{mv t10k\PYZhy{}images\PYZhy{}idx3\PYZhy{}ubyte.gz data/}\PY{l+s+s2}{\PYZdq{}}\PY{p}{)}
\end{Verbatim}


\begin{Verbatim}[commandchars=\\\{\}]
{\color{outcolor}Out[{\color{outcolor}23}]:} 0
\end{Verbatim}
            
    \begin{Verbatim}[commandchars=\\\{\}]
{\color{incolor}In [{\color{incolor}24}]:} \PY{c+c1}{\PYZsh{} !wget \PYZhy{}O data/t10k\PYZhy{}labels\PYZhy{}idx1\PYZhy{}ubyte.gz http://yann.lecun.com/exdb/mnist/t10k\PYZhy{}labels\PYZhy{}idx1\PYZhy{}ubyte.gz}
         \PY{n}{wget}\PY{o}{.}\PY{n}{download}\PY{p}{(}\PY{l+s+s2}{\PYZdq{}}\PY{l+s+s2}{http://yann.lecun.com/exdb/mnist/t10k\PYZhy{}labels\PYZhy{}idx1\PYZhy{}ubyte.gz}\PY{l+s+s2}{\PYZdq{}}\PY{p}{)}
         \PY{n}{os}\PY{o}{.}\PY{n}{system} \PY{p}{(}\PY{l+s+s2}{\PYZdq{}}\PY{l+s+s2}{mv t10k\PYZhy{}labels\PYZhy{}idx1\PYZhy{}ubyte.gz data/}\PY{l+s+s2}{\PYZdq{}}\PY{p}{)}
\end{Verbatim}


\begin{Verbatim}[commandchars=\\\{\}]
{\color{outcolor}Out[{\color{outcolor}24}]:} 0
\end{Verbatim}
            
    If you cant unzip the file, you can try gzip or download it from
\href{https://github.com/mGalarnyk/Python_Tutorials/tree/master/Sklearn/Logistic_Regression/data}{my
github}

    \begin{Verbatim}[commandchars=\\\{\}]
{\color{incolor}In [{\color{incolor}25}]:} \PY{c+c1}{\PYZsh{} decompress gzipped file}
         \PY{c+c1}{\PYZsh{}!info gzip}
         \PY{n}{os}\PY{o}{.}\PY{n}{system}\PY{p}{(}\PY{l+s+s2}{\PYZdq{}}\PY{l+s+s2}{gzip \PYZhy{}d data/*.gz}\PY{l+s+s2}{\PYZdq{}}\PY{p}{)}
\end{Verbatim}


\begin{Verbatim}[commandchars=\\\{\}]
{\color{outcolor}Out[{\color{outcolor}25}]:} 0
\end{Verbatim}
            
    \subsection{Loading MNIST Dataset}\label{loading-mnist-dataset}

    \begin{Verbatim}[commandchars=\\\{\}]
{\color{incolor}In [{\color{incolor}26}]:} \PY{k}{def} \PY{n+nf}{loadmnist}\PY{p}{(}\PY{n}{imagefile}\PY{p}{,} \PY{n}{labelfile}\PY{p}{)}\PY{p}{:}
         
             \PY{c+c1}{\PYZsh{} Open the images with gzip in read binary mode}
             \PY{n}{images} \PY{o}{=} \PY{n+nb}{open}\PY{p}{(}\PY{n}{imagefile}\PY{p}{,} \PY{l+s+s1}{\PYZsq{}}\PY{l+s+s1}{rb}\PY{l+s+s1}{\PYZsq{}}\PY{p}{)}
             \PY{n}{labels} \PY{o}{=} \PY{n+nb}{open}\PY{p}{(}\PY{n}{labelfile}\PY{p}{,} \PY{l+s+s1}{\PYZsq{}}\PY{l+s+s1}{rb}\PY{l+s+s1}{\PYZsq{}}\PY{p}{)}
         
             \PY{c+c1}{\PYZsh{} Get metadata for images}
             \PY{n}{images}\PY{o}{.}\PY{n}{read}\PY{p}{(}\PY{l+m+mi}{4}\PY{p}{)}  \PY{c+c1}{\PYZsh{} skip the magic\PYZus{}number}
             \PY{n}{number\PYZus{}of\PYZus{}images} \PY{o}{=} \PY{n}{images}\PY{o}{.}\PY{n}{read}\PY{p}{(}\PY{l+m+mi}{4}\PY{p}{)}
             \PY{n}{number\PYZus{}of\PYZus{}images} \PY{o}{=} \PY{n}{unpack}\PY{p}{(}\PY{l+s+s1}{\PYZsq{}}\PY{l+s+s1}{\PYZgt{}I}\PY{l+s+s1}{\PYZsq{}}\PY{p}{,} \PY{n}{number\PYZus{}of\PYZus{}images}\PY{p}{)}\PY{p}{[}\PY{l+m+mi}{0}\PY{p}{]}
             \PY{n}{rows} \PY{o}{=} \PY{n}{images}\PY{o}{.}\PY{n}{read}\PY{p}{(}\PY{l+m+mi}{4}\PY{p}{)}
             \PY{n}{rows} \PY{o}{=} \PY{n}{unpack}\PY{p}{(}\PY{l+s+s1}{\PYZsq{}}\PY{l+s+s1}{\PYZgt{}I}\PY{l+s+s1}{\PYZsq{}}\PY{p}{,} \PY{n}{rows}\PY{p}{)}\PY{p}{[}\PY{l+m+mi}{0}\PY{p}{]}
             \PY{n}{cols} \PY{o}{=} \PY{n}{images}\PY{o}{.}\PY{n}{read}\PY{p}{(}\PY{l+m+mi}{4}\PY{p}{)}
             \PY{n}{cols} \PY{o}{=} \PY{n}{unpack}\PY{p}{(}\PY{l+s+s1}{\PYZsq{}}\PY{l+s+s1}{\PYZgt{}I}\PY{l+s+s1}{\PYZsq{}}\PY{p}{,} \PY{n}{cols}\PY{p}{)}\PY{p}{[}\PY{l+m+mi}{0}\PY{p}{]}
         
             \PY{c+c1}{\PYZsh{} Get metadata for labels}
             \PY{n}{labels}\PY{o}{.}\PY{n}{read}\PY{p}{(}\PY{l+m+mi}{4}\PY{p}{)}
             \PY{n}{N} \PY{o}{=} \PY{n}{labels}\PY{o}{.}\PY{n}{read}\PY{p}{(}\PY{l+m+mi}{4}\PY{p}{)}
             \PY{n}{N} \PY{o}{=} \PY{n}{unpack}\PY{p}{(}\PY{l+s+s1}{\PYZsq{}}\PY{l+s+s1}{\PYZgt{}I}\PY{l+s+s1}{\PYZsq{}}\PY{p}{,} \PY{n}{N}\PY{p}{)}\PY{p}{[}\PY{l+m+mi}{0}\PY{p}{]}
         
             \PY{c+c1}{\PYZsh{} Get data}
             \PY{n}{x} \PY{o}{=} \PY{n}{np}\PY{o}{.}\PY{n}{zeros}\PY{p}{(}\PY{p}{(}\PY{n}{N}\PY{p}{,} \PY{n}{rows}\PY{o}{*}\PY{n}{cols}\PY{p}{)}\PY{p}{,} \PY{n}{dtype}\PY{o}{=}\PY{n}{np}\PY{o}{.}\PY{n}{uint8}\PY{p}{)}  \PY{c+c1}{\PYZsh{} Initialize numpy array}
             \PY{n}{y} \PY{o}{=} \PY{n}{np}\PY{o}{.}\PY{n}{zeros}\PY{p}{(}\PY{n}{N}\PY{p}{,} \PY{n}{dtype}\PY{o}{=}\PY{n}{np}\PY{o}{.}\PY{n}{uint8}\PY{p}{)}  \PY{c+c1}{\PYZsh{} Initialize numpy array}
             \PY{k}{for} \PY{n}{i} \PY{o+ow}{in} \PY{n+nb}{range}\PY{p}{(}\PY{n}{N}\PY{p}{)}\PY{p}{:}
                 \PY{k}{for} \PY{n}{j} \PY{o+ow}{in} \PY{n+nb}{range}\PY{p}{(}\PY{n}{rows}\PY{o}{*}\PY{n}{cols}\PY{p}{)}\PY{p}{:}
                     \PY{n}{tmp\PYZus{}pixel} \PY{o}{=} \PY{n}{images}\PY{o}{.}\PY{n}{read}\PY{p}{(}\PY{l+m+mi}{1}\PY{p}{)}  \PY{c+c1}{\PYZsh{} Just a single byte}
                     \PY{n}{tmp\PYZus{}pixel} \PY{o}{=} \PY{n}{unpack}\PY{p}{(}\PY{l+s+s1}{\PYZsq{}}\PY{l+s+s1}{\PYZgt{}B}\PY{l+s+s1}{\PYZsq{}}\PY{p}{,} \PY{n}{tmp\PYZus{}pixel}\PY{p}{)}\PY{p}{[}\PY{l+m+mi}{0}\PY{p}{]}
                     \PY{n}{x}\PY{p}{[}\PY{n}{i}\PY{p}{]}\PY{p}{[}\PY{n}{j}\PY{p}{]} \PY{o}{=} \PY{n}{tmp\PYZus{}pixel}
                 \PY{n}{tmp\PYZus{}label} \PY{o}{=} \PY{n}{labels}\PY{o}{.}\PY{n}{read}\PY{p}{(}\PY{l+m+mi}{1}\PY{p}{)}
                 \PY{n}{y}\PY{p}{[}\PY{n}{i}\PY{p}{]} \PY{o}{=} \PY{n}{unpack}\PY{p}{(}\PY{l+s+s1}{\PYZsq{}}\PY{l+s+s1}{\PYZgt{}B}\PY{l+s+s1}{\PYZsq{}}\PY{p}{,} \PY{n}{tmp\PYZus{}label}\PY{p}{)}\PY{p}{[}\PY{l+m+mi}{0}\PY{p}{]}
         
             \PY{n}{images}\PY{o}{.}\PY{n}{close}\PY{p}{(}\PY{p}{)}
             \PY{n}{labels}\PY{o}{.}\PY{n}{close}\PY{p}{(}\PY{p}{)}
             \PY{k}{return} \PY{p}{(}\PY{n}{x}\PY{p}{,} \PY{n}{y}\PY{p}{)}
\end{Verbatim}


    \begin{Verbatim}[commandchars=\\\{\}]
{\color{incolor}In [{\color{incolor}27}]:} \PY{n}{train\PYZus{}img}\PY{p}{,} \PY{n}{train\PYZus{}lbl} \PY{o}{=} \PY{n}{loadmnist}\PY{p}{(}\PY{l+s+s1}{\PYZsq{}}\PY{l+s+s1}{data/train\PYZhy{}images\PYZhy{}idx3\PYZhy{}ubyte}\PY{l+s+s1}{\PYZsq{}}
                                          \PY{p}{,} \PY{l+s+s1}{\PYZsq{}}\PY{l+s+s1}{data/train\PYZhy{}labels\PYZhy{}idx1\PYZhy{}ubyte}\PY{l+s+s1}{\PYZsq{}}\PY{p}{)}
         \PY{n}{test\PYZus{}img}\PY{p}{,} \PY{n}{test\PYZus{}lbl} \PY{o}{=} \PY{n}{loadmnist}\PY{p}{(}\PY{l+s+s1}{\PYZsq{}}\PY{l+s+s1}{data/t10k\PYZhy{}images\PYZhy{}idx3\PYZhy{}ubyte}\PY{l+s+s1}{\PYZsq{}}
                                        \PY{p}{,} \PY{l+s+s1}{\PYZsq{}}\PY{l+s+s1}{data/t10k\PYZhy{}labels\PYZhy{}idx1\PYZhy{}ubyte}\PY{l+s+s1}{\PYZsq{}}\PY{p}{)}
\end{Verbatim}


    \begin{Verbatim}[commandchars=\\\{\}]
{\color{incolor}In [{\color{incolor}28}]:} \PY{k}{print}\PY{p}{(}\PY{n}{train\PYZus{}img}\PY{o}{.}\PY{n}{shape}\PY{p}{)}
\end{Verbatim}


    \begin{Verbatim}[commandchars=\\\{\}]
(60000, 784)

    \end{Verbatim}

    \begin{Verbatim}[commandchars=\\\{\}]
{\color{incolor}In [{\color{incolor}29}]:} \PY{k}{print}\PY{p}{(}\PY{n}{train\PYZus{}lbl}\PY{o}{.}\PY{n}{shape}\PY{p}{)}
\end{Verbatim}


    \begin{Verbatim}[commandchars=\\\{\}]
(60000,)

    \end{Verbatim}

    \begin{Verbatim}[commandchars=\\\{\}]
{\color{incolor}In [{\color{incolor}30}]:} \PY{k}{print}\PY{p}{(}\PY{n}{test\PYZus{}img}\PY{o}{.}\PY{n}{shape}\PY{p}{)}
\end{Verbatim}


    \begin{Verbatim}[commandchars=\\\{\}]
(10000, 784)

    \end{Verbatim}

    \begin{Verbatim}[commandchars=\\\{\}]
{\color{incolor}In [{\color{incolor}31}]:} \PY{k}{print}\PY{p}{(}\PY{n}{test\PYZus{}lbl}\PY{o}{.}\PY{n}{shape}\PY{p}{)}
\end{Verbatim}


    \begin{Verbatim}[commandchars=\\\{\}]
(10000,)

    \end{Verbatim}

    \subsection{Showing Training Digits and
Labels}\label{showing-training-digits-and-labels}

    \begin{Verbatim}[commandchars=\\\{\}]
{\color{incolor}In [{\color{incolor}32}]:} \PY{n}{plt}\PY{o}{.}\PY{n}{figure}\PY{p}{(}\PY{n}{figsize}\PY{o}{=}\PY{p}{(}\PY{l+m+mi}{20}\PY{p}{,}\PY{l+m+mi}{4}\PY{p}{)}\PY{p}{)}
         \PY{k}{for} \PY{n}{index}\PY{p}{,} \PY{p}{(}\PY{n}{image}\PY{p}{,} \PY{n}{label}\PY{p}{)} \PY{o+ow}{in} \PY{n+nb}{enumerate}\PY{p}{(}\PY{n+nb}{zip}\PY{p}{(}\PY{n}{train\PYZus{}img}\PY{p}{[}\PY{l+m+mi}{0}\PY{p}{:}\PY{l+m+mi}{5}\PY{p}{]}\PY{p}{,} \PY{n}{train\PYZus{}lbl}\PY{p}{[}\PY{l+m+mi}{0}\PY{p}{:}\PY{l+m+mi}{5}\PY{p}{]}\PY{p}{)}\PY{p}{)}\PY{p}{:}
             \PY{n}{plt}\PY{o}{.}\PY{n}{subplot}\PY{p}{(}\PY{l+m+mi}{1}\PY{p}{,} \PY{l+m+mi}{5}\PY{p}{,} \PY{n}{index} \PY{o}{+} \PY{l+m+mi}{1}\PY{p}{)}
             \PY{n}{plt}\PY{o}{.}\PY{n}{imshow}\PY{p}{(}\PY{n}{np}\PY{o}{.}\PY{n}{reshape}\PY{p}{(}\PY{n}{image}\PY{p}{,} \PY{p}{(}\PY{l+m+mi}{28}\PY{p}{,}\PY{l+m+mi}{28}\PY{p}{)}\PY{p}{)}\PY{p}{,} \PY{n}{cmap}\PY{o}{=}\PY{n}{plt}\PY{o}{.}\PY{n}{cm}\PY{o}{.}\PY{n}{gray}\PY{p}{)}
             \PY{n}{plt}\PY{o}{.}\PY{n}{title}\PY{p}{(}\PY{l+s+s1}{\PYZsq{}}\PY{l+s+s1}{Training: }\PY{l+s+si}{\PYZpc{}i}\PY{l+s+se}{\PYZbs{}n}\PY{l+s+s1}{\PYZsq{}} \PY{o}{\PYZpc{}} \PY{n}{label}\PY{p}{,} \PY{n}{fontsize} \PY{o}{=} \PY{l+m+mi}{20}\PY{p}{)}
\end{Verbatim}


    \begin{center}
    \adjustimage{max size={0.9\linewidth}{0.9\paperheight}}{output_22_0.png}
    \end{center}
    { \hspace*{\fill} \\}
    
    \begin{Verbatim}[commandchars=\\\{\}]
{\color{incolor}In [{\color{incolor}33}]:} \PY{c+c1}{\PYZsh{} This is how the computer sees the number 5}
         \PY{k}{print}\PY{p}{(}\PY{n}{train\PYZus{}img}\PY{p}{[}\PY{l+m+mi}{0}\PY{p}{]}\PY{p}{)}
\end{Verbatim}


    \begin{Verbatim}[commandchars=\\\{\}]
[  0   0   0   0   0   0   0   0   0   0   0   0   0   0   0   0   0   0
   0   0   0   0   0   0   0   0   0   0   0   0   0   0   0   0   0   0
   0   0   0   0   0   0   0   0   0   0   0   0   0   0   0   0   0   0
   0   0   0   0   0   0   0   0   0   0   0   0   0   0   0   0   0   0
   0   0   0   0   0   0   0   0   0   0   0   0   0   0   0   0   0   0
   0   0   0   0   0   0   0   0   0   0   0   0   0   0   0   0   0   0
   0   0   0   0   0   0   0   0   0   0   0   0   0   0   0   0   0   0
   0   0   0   0   0   0   0   0   0   0   0   0   0   0   0   0   0   0
   0   0   0   0   0   0   0   0   3  18  18  18 126 136 175  26 166 255
 247 127   0   0   0   0   0   0   0   0   0   0   0   0  30  36  94 154
 170 253 253 253 253 253 225 172 253 242 195  64   0   0   0   0   0   0
   0   0   0   0   0  49 238 253 253 253 253 253 253 253 253 251  93  82
  82  56  39   0   0   0   0   0   0   0   0   0   0   0   0  18 219 253
 253 253 253 253 198 182 247 241   0   0   0   0   0   0   0   0   0   0
   0   0   0   0   0   0   0   0  80 156 107 253 253 205  11   0  43 154
   0   0   0   0   0   0   0   0   0   0   0   0   0   0   0   0   0   0
   0  14   1 154 253  90   0   0   0   0   0   0   0   0   0   0   0   0
   0   0   0   0   0   0   0   0   0   0   0   0   0 139 253 190   2   0
   0   0   0   0   0   0   0   0   0   0   0   0   0   0   0   0   0   0
   0   0   0   0   0  11 190 253  70   0   0   0   0   0   0   0   0   0
   0   0   0   0   0   0   0   0   0   0   0   0   0   0   0   0  35 241
 225 160 108   1   0   0   0   0   0   0   0   0   0   0   0   0   0   0
   0   0   0   0   0   0   0   0   0  81 240 253 253 119  25   0   0   0
   0   0   0   0   0   0   0   0   0   0   0   0   0   0   0   0   0   0
   0   0  45 186 253 253 150  27   0   0   0   0   0   0   0   0   0   0
   0   0   0   0   0   0   0   0   0   0   0   0   0  16  93 252 253 187
   0   0   0   0   0   0   0   0   0   0   0   0   0   0   0   0   0   0
   0   0   0   0   0   0   0 249 253 249  64   0   0   0   0   0   0   0
   0   0   0   0   0   0   0   0   0   0   0   0   0   0  46 130 183 253
 253 207   2   0   0   0   0   0   0   0   0   0   0   0   0   0   0   0
   0   0   0   0  39 148 229 253 253 253 250 182   0   0   0   0   0   0
   0   0   0   0   0   0   0   0   0   0   0   0  24 114 221 253 253 253
 253 201  78   0   0   0   0   0   0   0   0   0   0   0   0   0   0   0
   0   0  23  66 213 253 253 253 253 198  81   2   0   0   0   0   0   0
   0   0   0   0   0   0   0   0   0   0  18 171 219 253 253 253 253 195
  80   9   0   0   0   0   0   0   0   0   0   0   0   0   0   0   0   0
  55 172 226 253 253 253 253 244 133  11   0   0   0   0   0   0   0   0
   0   0   0   0   0   0   0   0   0   0 136 253 253 253 212 135 132  16
   0   0   0   0   0   0   0   0   0   0   0   0   0   0   0   0   0   0
   0   0   0   0   0   0   0   0   0   0   0   0   0   0   0   0   0   0
   0   0   0   0   0   0   0   0   0   0   0   0   0   0   0   0   0   0
   0   0   0   0   0   0   0   0   0   0   0   0   0   0   0   0   0   0
   0   0   0   0   0   0   0   0   0   0   0   0   0   0   0   0   0   0
   0   0   0   0   0   0   0   0   0   0]

    \end{Verbatim}

    \subsection{Using Logistic Regression on Entire
Dataset}\label{using-logistic-regression-on-entire-dataset}

    \href{http://scikit-learn.org/stable/modules/generated/sklearn.linear_model.LogisticRegression.html}{Logistic
Regression Sklearn Documentation} One thing I like to mention is the
importance of parameter tuning. While it may not have mattered much for
the toy digits dataset, it can make a major difference on larger and
more complex datasets you have. Please see the parameter: solver

    Step 1: Import the model you want to use

    In sklearn, all machine learning models are implemented as Python
classes

    \begin{Verbatim}[commandchars=\\\{\}]
{\color{incolor}In [{\color{incolor}34}]:} \PY{k+kn}{from} \PY{n+nn}{sklearn.linear\PYZus{}model} \PY{k+kn}{import} \PY{n}{LogisticRegression} 
\end{Verbatim}


    Step 2: Make an instance of the Model

    \begin{Verbatim}[commandchars=\\\{\}]
{\color{incolor}In [{\color{incolor}35}]:} \PY{c+c1}{\PYZsh{} all parameters not specified are set to their defaults}
         \PY{c+c1}{\PYZsh{} default solver is incredibly slow thats why we change it}
         \PY{n}{logisticRegr} \PY{o}{=} \PY{n}{LogisticRegression}\PY{p}{(}\PY{n}{solver} \PY{o}{=} \PY{l+s+s1}{\PYZsq{}}\PY{l+s+s1}{lbfgs}\PY{l+s+s1}{\PYZsq{}}\PY{p}{)}
\end{Verbatim}


    Step 3: Training the model on the data, storing the information learned
from the data

    Model is learning the relationship between x (digits) and y (labels)

    \begin{Verbatim}[commandchars=\\\{\}]
{\color{incolor}In [{\color{incolor}36}]:} \PY{n}{logisticRegr}\PY{o}{.}\PY{n}{fit}\PY{p}{(}\PY{n}{train\PYZus{}img}\PY{p}{,} \PY{n}{train\PYZus{}lbl}\PY{p}{)}
\end{Verbatim}


\begin{Verbatim}[commandchars=\\\{\}]
{\color{outcolor}Out[{\color{outcolor}36}]:} LogisticRegression(C=1.0, class\_weight=None, dual=False, fit\_intercept=True,
                   intercept\_scaling=1, max\_iter=100, multi\_class='ovr', n\_jobs=1,
                   penalty='l2', random\_state=None, solver='lbfgs', tol=0.0001,
                   verbose=0, warm\_start=False)
\end{Verbatim}
            
    Step 4: Predict the labels of new data (new images)

    Uses the information the model learned during the model training process

    \begin{Verbatim}[commandchars=\\\{\}]
{\color{incolor}In [{\color{incolor}37}]:} \PY{c+c1}{\PYZsh{} Returns a NumPy Array}
         \PY{c+c1}{\PYZsh{} Predict for One Observation (image)}
         \PY{n}{logisticRegr}\PY{o}{.}\PY{n}{predict}\PY{p}{(}\PY{n}{test\PYZus{}img}\PY{p}{[}\PY{l+m+mi}{0}\PY{p}{]}\PY{o}{.}\PY{n}{reshape}\PY{p}{(}\PY{l+m+mi}{1}\PY{p}{,}\PY{o}{\PYZhy{}}\PY{l+m+mi}{1}\PY{p}{)}\PY{p}{)}
\end{Verbatim}


\begin{Verbatim}[commandchars=\\\{\}]
{\color{outcolor}Out[{\color{outcolor}37}]:} array([7], dtype=uint8)
\end{Verbatim}
            
    \begin{Verbatim}[commandchars=\\\{\}]
{\color{incolor}In [{\color{incolor}38}]:} \PY{c+c1}{\PYZsh{} Predict for Multiple Observations (images) at Once}
         \PY{n}{logisticRegr}\PY{o}{.}\PY{n}{predict}\PY{p}{(}\PY{n}{test\PYZus{}img}\PY{p}{[}\PY{l+m+mi}{0}\PY{p}{:}\PY{l+m+mi}{10}\PY{p}{]}\PY{p}{)}
\end{Verbatim}


\begin{Verbatim}[commandchars=\\\{\}]
{\color{outcolor}Out[{\color{outcolor}38}]:} array([7, 2, 1, 0, 4, 1, 4, 9, 6, 9], dtype=uint8)
\end{Verbatim}
            
    \subsection{Measuring Model
Performance}\label{measuring-model-performance}

    accuracy (fraction of correct predictions): correct predictions / total
number of data points

    Basically, how the model performs on new data (test set)

    \begin{Verbatim}[commandchars=\\\{\}]
{\color{incolor}In [{\color{incolor}39}]:} \PY{n}{score} \PY{o}{=} \PY{n}{logisticRegr}\PY{o}{.}\PY{n}{score}\PY{p}{(}\PY{n}{test\PYZus{}img}\PY{p}{,} \PY{n}{test\PYZus{}lbl}\PY{p}{)}
         \PY{k}{print}\PY{p}{(}\PY{n}{score}\PY{p}{)}
\end{Verbatim}


    \begin{Verbatim}[commandchars=\\\{\}]
0.9176

    \end{Verbatim}

    \subsection{Confusion Matrix}\label{confusion-matrix}

    A confusion matrix is a table that is often used to describe the
performance of a classification model (or "classifier") on a set of test
data for which the true values are known.

    Note: Seaborn needs to be installed for this portion 

    \begin{Verbatim}[commandchars=\\\{\}]
{\color{incolor}In [{\color{incolor} }]:} \PY{c+c1}{\PYZsh{} !conda install seaborn \PYZhy{}y}
\end{Verbatim}


    \begin{Verbatim}[commandchars=\\\{\}]
{\color{incolor}In [{\color{incolor}40}]:} \PY{c+c1}{\PYZsh{} Make predictions on test data}
         \PY{n}{predictions} \PY{o}{=} \PY{n}{logisticRegr}\PY{o}{.}\PY{n}{predict}\PY{p}{(}\PY{n}{test\PYZus{}img}\PY{p}{)}
\end{Verbatim}


    \begin{Verbatim}[commandchars=\\\{\}]
{\color{incolor}In [{\color{incolor}41}]:} \PY{n}{cm} \PY{o}{=} \PY{n}{metrics}\PY{o}{.}\PY{n}{confusion\PYZus{}matrix}\PY{p}{(}\PY{n}{test\PYZus{}lbl}\PY{p}{,} \PY{n}{predictions}\PY{p}{)}
         \PY{n}{cm\PYZus{}normalized} \PY{o}{=} \PY{n}{cm}\PY{o}{.}\PY{n}{astype}\PY{p}{(}\PY{l+s+s1}{\PYZsq{}}\PY{l+s+s1}{float}\PY{l+s+s1}{\PYZsq{}}\PY{p}{)} \PY{o}{/} \PY{n}{cm}\PY{o}{.}\PY{n}{sum}\PY{p}{(}\PY{n}{axis}\PY{o}{=}\PY{l+m+mi}{1}\PY{p}{)}\PY{p}{[}\PY{p}{:}\PY{p}{,} \PY{n}{np}\PY{o}{.}\PY{n}{newaxis}\PY{p}{]}
\end{Verbatim}


    \begin{Verbatim}[commandchars=\\\{\}]
{\color{incolor}In [{\color{incolor}42}]:} \PY{n}{plt}\PY{o}{.}\PY{n}{figure}\PY{p}{(}\PY{n}{figsize}\PY{o}{=}\PY{p}{(}\PY{l+m+mi}{9}\PY{p}{,}\PY{l+m+mi}{9}\PY{p}{)}\PY{p}{)}
         \PY{n}{sns}\PY{o}{.}\PY{n}{heatmap}\PY{p}{(}\PY{n}{cm\PYZus{}normalized}\PY{p}{,} \PY{n}{annot}\PY{o}{=}\PY{n+nb+bp}{True}\PY{p}{,} \PY{n}{fmt}\PY{o}{=}\PY{l+s+s2}{\PYZdq{}}\PY{l+s+s2}{.3f}\PY{l+s+s2}{\PYZdq{}}\PY{p}{,} \PY{n}{linewidths}\PY{o}{=}\PY{o}{.}\PY{l+m+mi}{5}\PY{p}{,} \PY{n}{square} \PY{o}{=} \PY{n+nb+bp}{True}\PY{p}{,} \PY{n}{cmap} \PY{o}{=} \PY{l+s+s1}{\PYZsq{}}\PY{l+s+s1}{Blues\PYZus{}r}\PY{l+s+s1}{\PYZsq{}}\PY{p}{)}\PY{p}{;}
         \PY{n}{plt}\PY{o}{.}\PY{n}{ylabel}\PY{p}{(}\PY{l+s+s1}{\PYZsq{}}\PY{l+s+s1}{Actual label}\PY{l+s+s1}{\PYZsq{}}\PY{p}{)}\PY{p}{;}
         \PY{n}{plt}\PY{o}{.}\PY{n}{xlabel}\PY{p}{(}\PY{l+s+s1}{\PYZsq{}}\PY{l+s+s1}{Predicted label}\PY{l+s+s1}{\PYZsq{}}\PY{p}{)}\PY{p}{;}
         \PY{n}{all\PYZus{}sample\PYZus{}title} \PY{o}{=} \PY{l+s+s1}{\PYZsq{}}\PY{l+s+s1}{Accuracy Score: \PYZob{}:.3f\PYZcb{}}\PY{l+s+s1}{\PYZsq{}}\PY{o}{.}\PY{n}{format}\PY{p}{(}\PY{n}{score}\PY{p}{)} 
         \PY{n}{plt}\PY{o}{.}\PY{n}{title}\PY{p}{(}\PY{n}{all\PYZus{}sample\PYZus{}title}\PY{p}{,} \PY{n}{size} \PY{o}{=} \PY{l+m+mi}{15}\PY{p}{)}\PY{p}{;}
\end{Verbatim}


    \begin{center}
    \adjustimage{max size={0.9\linewidth}{0.9\paperheight}}{output_48_0.png}
    \end{center}
    { \hspace*{\fill} \\}
    
    \subsection{Display Misclassified images with Predicted
Labels}\label{display-misclassified-images-with-predicted-labels}

    \begin{Verbatim}[commandchars=\\\{\}]
{\color{incolor}In [{\color{incolor}43}]:} \PY{n}{index} \PY{o}{=} \PY{l+m+mi}{0}
         \PY{n}{misclassifiedIndexes} \PY{o}{=} \PY{p}{[}\PY{p}{]}
         \PY{k}{for} \PY{n}{label}\PY{p}{,} \PY{n}{predict} \PY{o+ow}{in} \PY{n+nb}{zip}\PY{p}{(}\PY{n}{test\PYZus{}lbl}\PY{p}{,} \PY{n}{predictions}\PY{p}{)}\PY{p}{:}
             \PY{k}{if} \PY{n}{label} \PY{o}{!=} \PY{n}{predict}\PY{p}{:} 
                 \PY{n}{misclassifiedIndexes}\PY{o}{.}\PY{n}{append}\PY{p}{(}\PY{n}{index}\PY{p}{)}
             \PY{n}{index} \PY{o}{+}\PY{o}{=}\PY{l+m+mi}{1}
\end{Verbatim}


    \begin{Verbatim}[commandchars=\\\{\}]
{\color{incolor}In [{\color{incolor}44}]:} \PY{n}{plt}\PY{o}{.}\PY{n}{figure}\PY{p}{(}\PY{n}{figsize}\PY{o}{=}\PY{p}{(}\PY{l+m+mi}{20}\PY{p}{,}\PY{l+m+mi}{4}\PY{p}{)}\PY{p}{)}
         \PY{k}{for} \PY{n}{plotIndex}\PY{p}{,} \PY{n}{badIndex} \PY{o+ow}{in} \PY{n+nb}{enumerate}\PY{p}{(}\PY{n}{misclassifiedIndexes}\PY{p}{[}\PY{l+m+mi}{0}\PY{p}{:}\PY{l+m+mi}{5}\PY{p}{]}\PY{p}{)}\PY{p}{:}
             \PY{n}{plt}\PY{o}{.}\PY{n}{subplot}\PY{p}{(}\PY{l+m+mi}{1}\PY{p}{,} \PY{l+m+mi}{5}\PY{p}{,} \PY{n}{plotIndex} \PY{o}{+} \PY{l+m+mi}{1}\PY{p}{)}
             \PY{n}{plt}\PY{o}{.}\PY{n}{imshow}\PY{p}{(}\PY{n}{np}\PY{o}{.}\PY{n}{reshape}\PY{p}{(}\PY{n}{test\PYZus{}img}\PY{p}{[}\PY{n}{badIndex}\PY{p}{]}\PY{p}{,} \PY{p}{(}\PY{l+m+mi}{28}\PY{p}{,}\PY{l+m+mi}{28}\PY{p}{)}\PY{p}{)}\PY{p}{,} \PY{n}{cmap}\PY{o}{=}\PY{n}{plt}\PY{o}{.}\PY{n}{cm}\PY{o}{.}\PY{n}{gray}\PY{p}{)}
             \PY{n}{plt}\PY{o}{.}\PY{n}{title}\PY{p}{(}\PY{l+s+s1}{\PYZsq{}}\PY{l+s+s1}{Predicted: \PYZob{}\PYZcb{}, Actual: \PYZob{}\PYZcb{}}\PY{l+s+s1}{\PYZsq{}}\PY{o}{.}\PY{n}{format}\PY{p}{(}\PY{n}{predictions}\PY{p}{[}\PY{n}{badIndex}\PY{p}{]}\PY{p}{,} \PY{n}{test\PYZus{}lbl}\PY{p}{[}\PY{n}{badIndex}\PY{p}{]}\PY{p}{)}\PY{p}{,} \PY{n}{fontsize} \PY{o}{=} \PY{l+m+mi}{20}\PY{p}{)}
\end{Verbatim}


    \begin{center}
    \adjustimage{max size={0.9\linewidth}{0.9\paperheight}}{output_51_0.png}
    \end{center}
    { \hspace*{\fill} \\}
    
    \subsection{Checking Performance Based on Training Set
Size}\label{checking-performance-based-on-training-set-size}

    A confusion matrix is a table that is often used to describe the
performance of a classification model (or "classifier") on a set of test
data for which the true values are known.

    \begin{Verbatim}[commandchars=\\\{\}]
{\color{incolor}In [{\color{incolor}45}]:} \PY{n}{regr} \PY{o}{=} \PY{n}{LogisticRegression}\PY{p}{(}\PY{n}{solver} \PY{o}{=} \PY{l+s+s1}{\PYZsq{}}\PY{l+s+s1}{lbfgs}\PY{l+s+s1}{\PYZsq{}}\PY{p}{)}
\end{Verbatim}


    \begin{Verbatim}[commandchars=\\\{\}]
{\color{incolor}In [{\color{incolor}46}]:} \PY{n}{fig}\PY{p}{,} \PY{n}{axes} \PY{o}{=} \PY{n}{plt}\PY{o}{.}\PY{n}{subplots}\PY{p}{(}\PY{n}{nrows} \PY{o}{=} \PY{l+m+mi}{1}\PY{p}{,} \PY{n}{ncols} \PY{o}{=} \PY{l+m+mi}{3}\PY{p}{,} \PY{n}{figsize} \PY{o}{=} \PY{p}{(}\PY{l+m+mi}{24}\PY{p}{,}\PY{l+m+mi}{8}\PY{p}{)}\PY{p}{)}\PY{p}{;}
         \PY{n}{plt}\PY{o}{.}\PY{n}{tight\PYZus{}layout}\PY{p}{(}\PY{p}{)}
         
         \PY{k}{for} \PY{n}{plotIndex}\PY{p}{,} \PY{n}{sample\PYZus{}size} \PY{o+ow}{in} \PY{n+nb}{enumerate}\PY{p}{(}\PY{p}{[}\PY{l+m+mi}{100}\PY{p}{,} \PY{l+m+mi}{1000}\PY{p}{,} \PY{l+m+mi}{60000}\PY{p}{]}\PY{p}{)}\PY{p}{:}
             \PY{n}{X\PYZus{}train} \PY{o}{=} \PY{n}{train\PYZus{}img}\PY{p}{[}\PY{p}{:}\PY{n}{sample\PYZus{}size}\PY{p}{]}\PY{o}{.}\PY{n}{reshape}\PY{p}{(}\PY{n}{sample\PYZus{}size}\PY{p}{,} \PY{l+m+mi}{784}\PY{p}{)}
             \PY{n}{y\PYZus{}train} \PY{o}{=} \PY{n}{train\PYZus{}lbl}\PY{p}{[}\PY{p}{:}\PY{n}{sample\PYZus{}size}\PY{p}{]}
             \PY{n}{regr}\PY{o}{.}\PY{n}{fit}\PY{p}{(}\PY{n}{X\PYZus{}train}\PY{p}{,} \PY{n}{y\PYZus{}train}\PY{p}{)}
             \PY{n}{predicted} \PY{o}{=} \PY{n}{regr}\PY{o}{.}\PY{n}{predict}\PY{p}{(}\PY{n}{test\PYZus{}img}\PY{p}{)}
             \PY{n}{cm} \PY{o}{=} \PY{n}{metrics}\PY{o}{.}\PY{n}{confusion\PYZus{}matrix}\PY{p}{(}\PY{n}{test\PYZus{}lbl}\PY{p}{,} \PY{n}{predicted}\PY{p}{)}
             \PY{n}{cm\PYZus{}normalized} \PY{o}{=} \PY{n}{cm}\PY{o}{.}\PY{n}{astype}\PY{p}{(}\PY{l+s+s1}{\PYZsq{}}\PY{l+s+s1}{float}\PY{l+s+s1}{\PYZsq{}}\PY{p}{)} \PY{o}{/} \PY{n}{cm}\PY{o}{.}\PY{n}{sum}\PY{p}{(}\PY{n}{axis}\PY{o}{=}\PY{l+m+mi}{1}\PY{p}{)}\PY{p}{[}\PY{p}{:}\PY{p}{,} \PY{n}{np}\PY{o}{.}\PY{n}{newaxis}\PY{p}{]}
             \PY{n}{sns}\PY{o}{.}\PY{n}{heatmap}\PY{p}{(}\PY{n}{cm\PYZus{}normalized}\PY{p}{,} \PY{n}{annot}\PY{o}{=}\PY{n+nb+bp}{True}\PY{p}{,} \PY{n}{fmt}\PY{o}{=}\PY{l+s+s2}{\PYZdq{}}\PY{l+s+s2}{.3f}\PY{l+s+s2}{\PYZdq{}}\PY{p}{,} \PY{n}{linewidths}\PY{o}{=}\PY{o}{.}\PY{l+m+mi}{5}\PY{p}{,} \PY{n}{square} \PY{o}{=} \PY{n+nb+bp}{True}\PY{p}{,} \PY{n}{cmap} \PY{o}{=} \PY{l+s+s1}{\PYZsq{}}\PY{l+s+s1}{Blues\PYZus{}r}\PY{l+s+s1}{\PYZsq{}}\PY{p}{,} \PY{n}{ax} \PY{o}{=} \PY{n}{axes}\PY{p}{[}\PY{n}{plotIndex}\PY{p}{]}\PY{p}{,} \PY{n}{cbar} \PY{o}{=} \PY{n+nb+bp}{False}\PY{p}{)}\PY{p}{;}
             \PY{n}{accuracyString} \PY{o}{=} \PY{l+s+s1}{\PYZsq{}}\PY{l+s+s1}{\PYZob{}:g\PYZcb{} Training Samples Score: \PYZob{}:.3f\PYZcb{}}\PY{l+s+s1}{\PYZsq{}}\PY{o}{.}\PY{n}{format}\PY{p}{(}\PY{n}{sample\PYZus{}size}\PY{p}{,} \PY{n}{regr}\PY{o}{.}\PY{n}{score}\PY{p}{(}\PY{n}{test\PYZus{}img}\PY{p}{,} \PY{n}{test\PYZus{}lbl}\PY{p}{)}\PY{p}{)} 
             \PY{n}{axes}\PY{p}{[}\PY{n}{plotIndex}\PY{p}{]}\PY{o}{.}\PY{n}{set\PYZus{}title}\PY{p}{(}\PY{n}{accuracyString}\PY{p}{,} \PY{n}{size} \PY{o}{=} \PY{l+m+mi}{25}\PY{p}{)}\PY{p}{;}
         
         \PY{n}{axes}\PY{p}{[}\PY{l+m+mi}{0}\PY{p}{]}\PY{o}{.}\PY{n}{set\PYZus{}ylabel}\PY{p}{(}\PY{l+s+s1}{\PYZsq{}}\PY{l+s+s1}{Actual label}\PY{l+s+s1}{\PYZsq{}}\PY{p}{,} \PY{n}{fontsize} \PY{o}{=} \PY{l+m+mi}{30}\PY{p}{)}\PY{p}{;}
         \PY{n}{axes}\PY{p}{[}\PY{l+m+mi}{1}\PY{p}{]}\PY{o}{.}\PY{n}{set\PYZus{}xlabel}\PY{p}{(}\PY{l+s+s1}{\PYZsq{}}\PY{l+s+s1}{Predicted label}\PY{l+s+s1}{\PYZsq{}}\PY{p}{,} \PY{n}{fontsize} \PY{o}{=} \PY{l+m+mi}{30}\PY{p}{)}\PY{p}{;}
\end{Verbatim}


    \begin{center}
    \adjustimage{max size={0.9\linewidth}{0.9\paperheight}}{output_55_0.png}
    \end{center}
    { \hspace*{\fill} \\}
    
    if this tutorial doesn't cover what you are looking for, please leave a
comment on the youtube video and I will try to cover what you are
interested in. 

    \href{https://www.youtube.com/watch?v=71iXeuKFcQM}{youtube video}


    % Add a bibliography block to the postdoc
    
    
    
    \end{document}
